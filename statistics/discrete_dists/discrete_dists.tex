\documentclass{article}
\usepackage{import}
\usepackage[english]{babel}
\usepackage[utf8]{inputenc}
\usepackage{amsmath,amssymb,amsthm}
\usepackage{verbatim}
\usepackage{enumitem}
\usepackage{graphicx}
\newtheorem{definition}{Definition}
\newtheorem{theorem}{Theorem}

\usepackage[top=2.5cm, left=3cm, right=3cm, bottom=4.0cm]{geometry}
\usepackage[table]{xcolor}

\newcommand{\tablespace}{\\[1.25mm]}
\newcommand\Tstrut{\rule{0pt}{2.6ex}}         % = `top' strut
\newcommand\tstrut{\rule{0pt}{2.0ex}}         % = `top' strut
\newcommand\Bstrut{\rule[-0.9ex]{0pt}{0pt}}   % = `bottom' strut

\title{Discrete Probability Distributions}
\date{21-12-2020}
\author{Anjali Bhavan}
\begin{document}
\maketitle
\section{Uniform distribution}
The uniform distribution is one in which every value that a random variable $ X $ can assume has equal
probability. That is, its p.d.f is given by:
\begin{equation*}
P(X = x_{i}) = \frac{1}{n}, i = 1, 2, ... n
\end{equation*}
The M.G.F about origin of a uniform distribution is given by:
\begin{equation*}
\begin{split}
M_{o}(t) & = E(e^{tX}) \\
  & = \frac{e^{t}(1-e^{nt})}{n(1-e^{t})}
\end{split}
\end{equation*}
\section{Binomial distribution}
Bernoulli trials are those in which there are only two possible outcomes (success and failure), and the probability of success remains constant throughout the trial.
Binomial distribution comes into play when we wish to find the probability of $ x $ successes in $ n $ Bernoulli trials. \\
If a RV $ X $ has probability mass function 
\begin{equation*}
p(x) = \binom{n}{x}p^{x} q^{n-x} 
\end{equation*}
It is a binomial variate and has binomial distribution. \\
Expectation of a binomial variate is given by $ np $. Variance is $ npq $. M.G.F about origin = $ (pe^{t}+q)^{n} $. M.G.F about the mean = $ e^{-npt}(pe^{t}+q)^{n} $.

\section{Geometric distribution}
This describes the probabilities of failure before the first success.
\begin{equation*}
p(x) = q^{x}p
\end{equation*}
Expectation of a geometric variate is given by $ \frac{q}{p} $. Variance is $ \frac{q}{p^{2}} $. M.G.F about origin = $ \frac{p}{1-qe^{t}} $. 

\section{Poisson distribution}
In the case of a binomial distribution, when $ n $ is large and $ p $ is small such that the average number of successes (expectation) $ np $ is a finite constant (say $ \lambda $ ),
the distribution is called Poisson and its probability function can be generated by substituting $ p = \frac{\lambda}{n} $ in the probability function of the binomial distribution, and taking limit as $ n \to \infty $ . 
\begin{equation*}
p(x) = \frac{e^{- \lambda} \lambda^{x}}{x!}, x = 0,1,2... \infty
\end{equation*}
In case of Poisson variate mean and variance are both equal to $ \lambda $. The M.G.F about origin is given by $ exp[\lambda(e^{t}-1)] $   
\end{document}