\documentclass{article}
\usepackage{import}
\usepackage[english]{babel}
\usepackage[utf8]{inputenc}
\usepackage{amsmath,amssymb,amsthm}
\usepackage{verbatim}
\usepackage{enumitem}
\usepackage{graphicx}
\newtheorem{definition}{Definition}
\newtheorem{theorem}{Theorem}

\usepackage[top=2.5cm, left=3cm, right=3cm, bottom=4.0cm]{geometry}
\usepackage[table]{xcolor}

\newcommand{\tablespace}{\\[1.25mm]}
\newcommand\Tstrut{\rule{0pt}{2.6ex}}         % = `top' strut
\newcommand\tstrut{\rule{0pt}{2.0ex}}         % = `top' strut
\newcommand\Bstrut{\rule[-0.9ex]{0pt}{0pt}}   % = `bottom' strut

\title{Random Variables and Expectation}
\date{20-12-2020}
\author{Anjali Bhavan}

\begin{document}
\maketitle

\section{Random variables}
\begin{definition}[Random variable]
    A function that assigns a real value $ X(w) $ to every outcome $ w $ in the sample space.
\end{definition}
Two types: discrete and continuous. 
\begin{enumerate}
  \item Distribution function for discrete RV: 
  \begin{equation*}
  F(x) = \sum_{i:x_{i} \leq x}^{} p(x_{i})
  \end{equation*}
  where $ p(x_{i}) $ is the probability mass function (basically probability value).
  \item Distribution function for continuous RV:
  \begin{equation*}
  F(x) = \int_{-\infty}^{x} f(x) \,dx 
  \end{equation*}
  where $ f(x) $ is the probability density function.
\end{enumerate}

\section{Jointly distributed random variables}
For two RVs $ X $ and $ Y $ associated with the same random experiment,
\begin{enumerate}
  \item when $ X $ and $ Y $ are discrete: \\
  Probability mass function:
  \begin{equation*}
    p(x_{i}, y_{i}) = P(X = x_{i}, Y = y_{i})
  \end{equation*} 
  and
  \begin{equation*}
    \sum_{x}^{} \sum_{y}^{} p(x_{i}, y_{j}) = 1 
  \end{equation*}
  Marginal probability mass functions are given by: 
  \begin{equation*}
    p(x_{i}) = P(X = x_{i}) = \sum_{j}^{}P(X = x_{i}, Y = y_{j}) = \sum_{j}^{}p_{ij}
  \end{equation*}
  \begin{equation*}
    p(y_{j}) = P(Y = y_{j}) = \sum_{i}^{}P(X = x_{i}, Y = y_{j}) = \sum_{i}^{}p_{ij}
  \end{equation*}

\item when $ X $ and $ Y $ are continuous: \\
The joint probability density function is given by $ f(x, y) $, and
\begin{equation*}
\int_{-\infty}^{\infty} \int_{-\infty}^{\infty} f(x,y) \,dx  \,dy = 1
\end{equation*}

The marginal p.d.fs are given by:
\begin{equation*}
f_{X}(x) = \int_{-\infty}^{\infty} f(x, y) \,dy
\end{equation*}
\begin{equation*}
    f_{Y}(y) = \int_{-\infty}^{\infty} f(x, y) \,dx
\end{equation*}

\textbf{Important:} if $ C = \{(x,y): a \leq x \leq b, c \leq y \leq d\} $, then
\begin{equation*}
P((X,Y) \in C) = \int_{a}^{b} \int_{c}^{d} f(x,y) \,dx  \,dy
\end{equation*}
\end{enumerate}
\section{Expectation}
\begin{definition}[Expectation]
    Expectation of a random variable $ X $ is the weighted average of all the possible values $ X $
    can take. For a discrete RV, it is given by:
    \begin{equation*}
    E(X) = \sum_{i}^{}p_{i}x_{i}
    \end{equation*}
    And for a continuous RV, 
    \begin{equation*}
    E(X) = \int_{-\infty}^{\infty}xf(x)  \,dx 
    \end{equation*}
\end{definition}

\textbf{Note:} Addition property of expectation is valid for any two random variables $X$ and $Y$, but multiplication
  property is valid only if $X$ and $Y$ are independent. Additive property states that $ E(X+Y) = E(X) + E(Y) $.
  Multiplication property states that $ E(XY) = E(X)E(Y) $ iff $X$ and $Y$ are independent.
\textbf{Note: } Variance is given by $ E(X^{2}) - E(X)^{2} $.

\subsection{Covariance}
\begin{definition}[Covariance]
    Covariance of two random variables $X$ and $Y$ is a measure of how the two are related to each other. It is given by:
    \begin{equation*}
        \begin{split}
            Cov(X, Y) & = E(X - \bar{X})(Y - \bar{Y}) \\
            & = E(XY - X\bar{Y} - Y\bar{X} + \bar{X}\bar{Y})  \\
            & = E(XY) - \bar{Y}E(X) - \bar{X}E(Y) + \bar{X}\bar{Y} \\
            & = E(XY) - E(X)E(Y)
        \end{split}
    \end{equation*}
\end{definition}
Positive covariance means $X$ and $Y$ are increasing or decreasing together, negative means the opposite.
When $X$ and $Y$ are independent, covariance is 0. But converse is not always true.

Some important formulae:
\begin{itemize}
  \item $ Var(aX+b) = a^{2}Var(X) $
  \item $ Var(aX+bY) = a^{2}Var(X) + b^{2}Var(Y) + 2abCov(X, Y) $
\end{itemize}

\section{Moment generating function}
The moment generating function of a RV $ X $ is given by: 
\begin{equation*}
M(X) = E[e^{tX}] = \begin{cases}
        \sum_{x}^{}e^{tx}p(x) & X is discrete \\
        \int_{-\infty}^{\infty} e^{tx}f(x)\,dx & X is continuous
    \end{cases}
\end{equation*}
Moments are usually measures for a function's graph shape. Particular moments have particular values: for instance,
the zeroth moment of a random variable is 1, the first the mean, the second the variance, the third the skewness and fourth the kurtosis.
MGF can describe a distribution uniquely if it exists. \\
If $X$ and $Y$ are two RVs, their sum's MGF is given by the product of the individual MGFs. \\
\textbf{Important:} The MGF of a RV about any arbitrary point can be derived from the MGF about origin by:
\begin{equation*}
M_{a}(t) = E[e^{t(x-a)}] = e^{-at}E[e^{tx}] = e^{-at}M_{o}(t)
\end{equation*}

\section{Chebyshev's Inequality}
This basically shows that no more than a certain fraction of values can lie more than a few standard deviations away from the mean.
Specifically, no more than $ 1/k^{2} $  of the distribution's values can be more than $ k $ standard deviations away from the mean.
Mathematically, let $ X $ be a random variable with mean $ \mu $ and variance $ \sigma^{2} $. Then,
\begin{equation*}
P(|X - \mu| \geq k\sigma) \leq \frac{1}{k^{2}}
\end{equation*}
This is used to prove the weak law of large numbers, which states:
\begin{theorem} The probability that the average of a sequence of i.i.d. RVs differs from its mean by more than $ \epsilon $ tends to 0 as 
  the number of terms tends to infinity, provided the RVs have finite variance.
\end{theorem}
\end{document}
