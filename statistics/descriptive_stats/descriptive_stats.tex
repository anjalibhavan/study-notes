\documentclass{article}
\usepackage{import}
\usepackage[english]{babel}
\usepackage[utf8]{inputenc}
\usepackage{amsmath,amssymb,amsthm}
\usepackage{verbatim}
\usepackage{enumitem}
\usepackage{graphicx}
\newtheorem{definition}{Definition}
\newtheorem{theorem}{Theorem}

\usepackage[top=2.5cm, left=3cm, right=3cm, bottom=4.0cm]{geometry}
\usepackage[table]{xcolor}

\newcommand{\tablespace}{\\[1.25mm]}
\newcommand\Tstrut{\rule{0pt}{2.6ex}}         % = `top' strut
\newcommand\tstrut{\rule{0pt}{2.0ex}}         % = `top' strut
\newcommand\Bstrut{\rule[-0.9ex]{0pt}{0pt}}   % = `bottom' strut

\title{Descriptive Statistics}
\date{19-12-2020}
\author{Anjali Bhavan}

\begin{document}
\maketitle
\section{Measures of central tendency}
\begin{enumerate}
    \item Arithmetic mean = $\frac{1}{N}\sum_{i = 1}^{n} f_{i}x_{i}$ 
    where $N = \sum_{i = 1}^{n} f_{i} $ \\
    Shifting of origin: $x = a + hu$, where $a$ is the shifted coordinate and $h$ is the new scale.
    \item Geometric mean 
    \item Harmonic mean
    \item Median = $l + \frac{h}{f}(\frac{N}{2}-C)$ \\
    where $l$ is the lower limit of the median class (class with cumulative frequency just more than $N/2$ (or
    whatever is the first value in the bracket)), $h$ and $f$ the width and frequency of median class,
    $C$ is cumulative frequency of pre-median class.
    \item Mode = $l + \frac{h(f_{m} - f_{1})}{2f_{m} - f_{1} - f_{2}}$ \\
    where $l$ is the lower limit of the modal class (class with highest frequency), $f1$ and $f2$ are
    frequencies of pre and post modal class.
    \item Partition values: \\
    Quartiles: $Q_{i} = l + \frac{h}{f}(\frac{iN}{4}-C)$ \\
    \textbf{Reminder: l, h, f and C are all to be figured out based on the first value in the bracket. That has to be
    used for determining which is median class.}
\end{enumerate}
\section{Measures of dispersion}
\begin{enumerate}
    \item Range: IQR = $Q3 - Q1$
    \item Std deviation: \textbf{Has two different formulae for population and sample.} Also,
    stddev is used rather than variance because square numbers get big, and variance has same unit as mean, hence. \\
    With shifting of origin: both variance and stddev are unaffected by origin shift, but are affected by 
    scale shift. Variance becomes $h^{2}$ times and stddev becomes $h$ times.
    \item Karl Pearson's coefficient of variation: used to measure variability in data (mostly for comparison). Given by $\frac{\sigma}{\bar{x}}*100$
    \item coefficient of dispersion based on quartile deviation: $\frac{Q3-Q1}{Q3+Q1}$

\end{enumerate}
\begin{definition}[Skewness]
    Skewness is the measure of lack of symmetry in the distribution. Positively skewed means the longer tail of the
    distribution is towards the right, negatively skewed means vice-versa. In positive skew, \textbf{mode $<$ median $<$ mean}. 
    In negative, \textbf{mean $<$ median $<$ mode}.
\end{definition}

\begin{definition}[Kurtosis]
    Kurtosis is the measure of lack of peakedness of the distribution. Leptokurtic distribution means more peaked than normal, mesokurtic
    means normal symmetric curve, platykurtic means flatter than normal curve. The value of $\beta_{2}$ (described below)
    gives an estimation of kurtosis.
\end{definition}

\section{Skewness, Kurtosis, Moments}
\begin{enumerate}
    \item rth moment of a distribution: given by 
    \begin{equation}
        \frac{1}{N}\sum_{i = 1}^{n}f_{i}(x_{i}-a)^r
    \end{equation}
    First moment is mean, second variance, third skewness, fourth kurtosis.
    \item Pearson's coefficients: \\
    $\beta_{1} = \frac{\mu_{3}^{2}}{\mu_{2}^{3}}$, $\gamma_{1} = \sqrt{\beta_{1}}$ \\
    $\beta_{2} = \frac{\mu_{4}}{\mu_{2}^{2}}$, $\gamma_{2} = \beta_{2} - 3$
    \item Karl Pearson's coefficient of skewness: $\frac{mean - mode}{\sigma}$
    \item Bowley's coefficient of skewness: $\frac{Q3 + Q1 - 2Q2}{Q3 - Q1}$
\end{enumerate}
\begin{theorem}
    Chebyshev's Inequality: Given a number $k \geq 1$ and a dataset of $n$ observations, 
    atleast $1 - \frac{1}{k^2}$ of the observations will lie within $k$ standard devs of the mean.

\end{theorem}

\end{document}