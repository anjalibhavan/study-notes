\documentclass{article}
\usepackage{import}
\usepackage[english]{babel}
\usepackage[utf8]{inputenc}
\usepackage{amsmath,amssymb,amsthm}
\usepackage{verbatim}
\usepackage{enumitem}
\usepackage{graphicx}
\newtheorem{definition}{Definition}
\newtheorem{theorem}{Theorem}

\usepackage[top=2.5cm, left=3cm, right=3cm, bottom=4.0cm]{geometry}
\usepackage[table]{xcolor}

\newcommand{\tablespace}{\\[1.25mm]}
\newcommand\Tstrut{\rule{0pt}{2.6ex}}         % = `top' strut
\newcommand\tstrut{\rule{0pt}{2.0ex}}         % = `top' strut
\newcommand\Bstrut{\rule[-0.9ex]{0pt}{0pt}}   % = `bottom' strut

\title{Basic Probability Theory}
\date{07-10-2020}
\author{Anjali Bhavan}

\begin{document}
\maketitle
\section{Basic concepts}
\begin{itemize}
    \item Some terms: possible outcomes, sample space etc.
    \item Random variables
    \item Cond. Probability
\end{itemize}
\section{Some Terms}
\begin{itemize}
    \item outcome: possible result of an experiment 
    \item sample space: set of all possible outcomes
    \item event: a set of outcomes of an experiment (subset of sample
    space)
    \item event space: set of all possible events 
    (power set of sample space?)
\end{itemize}
\section{Random Variables}
\begin{definition}[Random variables]
A random variable is a function from the sample space to R: $\Omega \rightarrow \mathbb{R}$ 
\end{definition}

There are two types of random variables: discrete and continuous.
\begin{itemize}
    \item Discrete:
    \begin{enumerate}[label=(\alph*)]
        \item Uniform: all outcomes have same probability
        \item Bernoulli: two possible outcomes
        \item Binomial: two possible outcomes, $n$ trials
        \item Multinomial: $k$ possible outcomes, $n$ trials
        \item Poisson: big $n$, small $p$ approx. of binomial
    \end{enumerate}
    \item Continuous:
    \begin{enumerate}[label=(\alph*)]
        \item continuous uniform distribution
        \item Gaussian (normal distribution)
    \end{enumerate}
\end{itemize}

\begin{definition}[Random vector]
Finite dimensional vector of random variables: 
$X = [X_{1},..., X_{k}]$.

$P(x) = P(x_{1},...,x_{n}) = P(X_{1} = x_{1},...X_{n} = x_{n}))$
\end{definition}
\section{Probability}
Three types:
\begin{itemize}
    \item Joint Probability: prob. of $X=x$ and $Y=y$ happening
    together
    \item Conditional Probability: prob. of $X=x$ given $Y=y$
    \item Marginal Probability: prob. of $X=x \forall Y$.
\end{itemize}

Chain rule: Calculate joint prob from marginal and condl. prob
\begin{equation}
    P(A,B) = P(A)*P(B|A) = P(B)*P(A|B)
\end{equation}

Calculating marginal prob from joint prob:
\begin{equation}
    P(A) = \sum_{B}P(A,B)
\end{equation}

Bayes' Rule: 
\begin{equation}
    P(B|A) = \frac{P(A,B)}{P(A)} = \frac{P(A|B)P(B)}{P(A)}
\end{equation}
$A$ will be the constant factor in the question i.e. whose prob doesnt
change.

Important Bayes rule eqn (used in machine translation):

(arg max over $y$ means whichever $y$ gives max value of expression)
\begin{equation}
    y^{*} = \arg \max P(y|x) \\
    = \arg \max \frac{P(x|y)P(y)}{P(x)} \\
    = \arg \max P(x|y)P(y)
\end{equation}
(in the third step $P(X)$ is removed because constant term. We are
interested in best value of $y$.)
\section{Independence}
 \begin{definition}[Conditional Independence]
     Once we know $C$, the value of $A$ doesn't affect $B$ and 
     vice versa.

     $P(A,B|C) = P(A|C)P(B|C)$

     $P(A|B,C) = P(A|C)$

     $P(B|A,C) = P(B|C)$
 \end{definition}

Why do we make independence assumption in language models despite 
it not being true? because space and time of model. it becomes large
and unwieldy.



\end{document}









