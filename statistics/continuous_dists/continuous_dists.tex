\documentclass{article}
\usepackage{import}
\usepackage[english]{babel}
\usepackage[utf8]{inputenc}
\usepackage{amsmath,amssymb,amsthm}
\usepackage{verbatim}
\usepackage{enumitem}
\usepackage{graphicx}
\newtheorem{definition}{Definition}
\newtheorem{theorem}{Theorem}

\usepackage[top=2.5cm, left=3cm, right=3cm, bottom=4.0cm]{geometry}
\usepackage[table]{xcolor}

\newcommand{\tablespace}{\\[1.25mm]}
\newcommand\Tstrut{\rule{0pt}{2.6ex}}         % = `top' strut
\newcommand\tstrut{\rule{0pt}{2.0ex}}         % = `top' strut
\newcommand\Bstrut{\rule[-0.9ex]{0pt}{0pt}}   % = `bottom' strut

\title{Continuous Probability Distributions}
\date{23-12-2020}
\author{Anjali Bhavan}
\begin{document}
\maketitle
\section{Continuous Uniform Distribution}
This distribution has constant probability in a closed interval. Its p.d.f is given by
\begin{equation*}
f(x) = \begin{cases}
    \frac{1}{b-a}, & a \leq x \leq b \\
    0, & otherwise
\end{cases}
\end{equation*}
\section{Normal distribution}
This is basically the most important distribution in the field of statistics. It can be defined as:
\begin{definition}[Normal variate]
    A continuous RV with two parameters $ \mu $ and $ \sigma $ with the p.d.f
    \begin{equation*}
    f(x) = \frac{1}{\sqrt{2 \pi} \sigma} e^{\frac{-1}{2}(\frac{x - \mu}{\sigma})^{2}}, -\infty < x < \infty
    \end{equation*}
    Is called a normal variate and the distribution is called normal distribution.
\end{definition}
The constants of normal distribution are as follows: \\
The mean is $ \mu $, the variance $ \sigma^{2} $, the odd moments about mean are 0 and the even moments are given by
\begin{equation*}
    \mu_{2n} = (2n-1) \sigma^{2} \mu_{2n-2}
\end{equation*}  
The M.G.F about origin is given by 
\begin{equation*}
M_{X}(t) = e^{\mu t + \frac{1}{2} \sigma^{2} t^{2}}
\end{equation*}
The sum of two independent normal variates is also a normal variate with mean and variance as the sums of the individual means and variances.

\subsection{Standard Normal Variate}
If $ X $ is a normally distributed RV with mean $ \mu $ and variance $ \sigma^{2} $, and we define $ Z = \frac{X - \mu}{\sigma} $, then $ Z $ is a normal variate
with mean 0 and variance 1. This is called the \textit{standard normal variate} and its p.d.f is given by
\begin{equation*}
f(x) = \frac{1}{\sqrt{2 \pi}} e^{-\frac{1}{2}z^{2}}, -\infty < z < \infty
\end{equation*}

\subsection{Area property of normal curve}
The probability of a normal variate lying between two values $ x_{1} and x_{2} $ is given by the area under the normal curve from $ x_{1} to x_{2} $. That is,
\begin{equation*}
P(x_{1} \leq X \leq x_{2}) = P(z_{2}) - P(z_{1})
\end{equation*}
where $ P(z) $ is the \textit{normal definite integral} and gives the area under standard normal curve between $ Z = 0 $ and $ Z = z $.  

\begin{theorem}
    68-95-99.7 rule: States that almost all values of a normal distribution lie within three standard deviations of the mean: 68\% lie within
    the first stddev, 95\% in the second, and 99.7\% in the third.
\end{theorem}

\section{Exponential distribution}
The exponential distribution is given by 
\begin{equation*}
f(x) = ae^{-ax}
\end{equation*}
The mean is given by $ \frac{1}{a} $, variance $ \frac{1}{a^{2}} $ and the M.G.F about origin is
\begin{equation*}
M_{X}(t) = \sum_{r=0}^{\infty}(\frac{t}{a})^{r}, t < a
\end{equation*}

\section{Weibull distribution}
The Weibull distribution is basically the exponential distribution with parameters $ \alpha $ and $ \beta $. It is given by
\begin{equation*}
f(x) = \begin{cases}
    \alpha \beta x^{\beta -1} e^{-\alpha x^{\beta}}, x >0, \alpha > 0, \beta > 0 \\
    0, otherwise
\end{cases}
\end{equation*}  
Setting $ \beta = 1 $ in this yields the exponential distribution. 

\end{document}